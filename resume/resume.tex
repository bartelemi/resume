\documentclass[12pt,english]{resume}

\pagenumbering{gobble}

\usepackage[%
    a4paper,%
	left=0.5in,%
	right=0.5in,%
	top=0.5in,%
	bottom=0.5in,%
	footskip=.25in%
]{geometry}

	
\begin{document}

	\name{Bartłomiej Szostek}

	\labelwithimage{../assets/phone.png}{+44 (0)7565224959}
	\labelwithimage{../assets/email.png}{szostek.bartek@gmail.com}
	\labelwithimage{../assets/linkedin.png}{linkedin.com/in/bartlomiej-szostek/}
	\labelwithimage{../assets/github.png}{github.com/bartelemii/}
	
	\section{Personal Statement}
		A challenge driven and innovative programmer, interested in overall computer
		science and software design. Exhibits strong analytical abilities and logical
		thinking reinforced by academic and work experience. Instinctive team worker,
		with creative ideas supported by technical knowledge. Software developer with
		five years’ experience, gained in one of the biggest Polish IT\&T companies.

	\section{Key Achievements}
		Awarded 14,000 GBP scholarship to Cranfield University in 2016/17 for Computational and Software Techniques in Engineering on the Software Engineering for Technical Computing option.
		Awarded 34,500 PLN scholarship for participation in project ZIP at Silesian University of Technology, Poland in years 2012-2016. Scholarship granted in recognition of academic achievements.
		Design and development of ONKO-SYS, one of largest national IT systems for genetic/proteomic scientists using modern tools and frameworks under the WASKO S.A. company.
		Participated in two mentoring projects in cooperation with Silesian University of Technology, as well as Bombardier Transportation Poland and EgzoTech companies.

	\section{Career}
		\datedsubsection{
			Limejump Ltd.: London, England - Senior Software Engineer
		}{May 2020 - Present}

		\datedsubsection{
			Limejump Ltd.: London, England - Full Stack Developer
		}{Mar 2018 - Apr 2020}
		
		\datedsubsection{
			Titian Software Ltd.: London, England – Software Engineer
		}{Aug 2017 – Mar 2018}
		Titian Software specialises in sample management systems for pharmaceutical and bio-science industries. The company delivers its key product – Mosaic, to the customers in Europe, the United States and the Far East.
		- Contributed to projects to integrate Mosaic software with hardware vendors’ APIs utilising WCF Web Services. Wrote simulator-like software to imitate the external hardware behaviour.
		- Coded desktop applications using C\#, WinForms, WPF and WiX Toolset as well as SQLite and Oracle technologies within a Scrum team.
		- Supported the software builds and improved code development processes as a member of internal Software Process Improvement Group. This included configuration of Team City servers, overseeing builds and enhancements in TFS integration.
		- Programmed, improved, integrated, tested and implemented software applications in line with defined application specifications, development plans and strict internal coding standards.
		
		\datedsubsection{
			WASKO S.A.: Gliwice, Poland - Part-time Web Developer
		}{Sep 2014 – Sep 2016}
		Wasko is one of the leading Polish IT\&T companies. Established in 1988 the Company has provided its solutions to middle and large enterprises, mainly in power \& energy, telecom and banking sectors as well as to public and governmental administration.
		- Coded web-based portal applications using HTML, JavaScript, CSS for frontend and C\# and SQL for backend within 3 – 20 people team.
		- Designed and created websites and CMS, meeting requirements of WCAG 2.0 technical standards, for hospitals, like Centre of Oncology in Gliwice and Orlowski Hospital in Warsaw.
		- Maintained and updated Microsoft SQL databases for company’s client; created MSSQL and PLSQL queries for testing and validations for two web applications.
		- Ported frontend of web-based application in a team of 6, from JSP to modern technologies using React.js.
		- Mentored 3 junior web developers and interns across multiple projects; was involved in technical meetings.
		- Analysed trending ECMAScript 6 as well as CSS3 and related frameworks for web user interfaces; structured architecture for front-end layer with React, as well as jQuery and Bootstrap libraries.
		- Designed and built prototype of cross-platform mobile application for physicians, visualising patients’ data, using Ionic and Angular open source frameworks with incorporation of ES6, HTML5 and CSS3.

	\section{Education}

		\datedsubsection{
			MSc in Computational \& Software Techniques in Engineering (Double Diploma),
			Software Techniques in Engineering: Cranfield University, Cranfield, UK
			(First Class Honours)
		}{Sep 2016 – Sep 2017}
			Modules includes: C++, Computational Methods, Requirement Analysis and System Design, Cloud Computing (AWS), Small Scale Parallel Programming (CUDA and OpenMP), Management for Technology.
			Group Design Project: Applications in Practical High-End Computing:
			Working in an international group of 5 people on simulation software as a programmer. Responsibilities included specifying overall program architecture, coding and managing project’s VS Team Services platform. Project involved improving easiness of usability, performance and architecture as well as extending features of software. The core of the application has strongly used CUDA technology.
			Individual Thesis: Optimised simulation of reduced aeroelastic systems:
			Project aimed to redesign a CA2LM framework able to simulate in approximation a flexible aircraft model of reduced complexity. Tool is strongly utilised by researchers at Cranfield University. Task required to explore scope for improvements in simulation run-times and design new engineering interfaces to improve learning curve of the framework. Final results yielded 62\% speed improvements, new user-friendly interface and modular, maintainable software, while preserving existing framework capabilities.

		\datedsubsection{
			MSc in Informatics (Double Diploma): Silesian University of Technology,
			Faculty of Automatic Control, Electronics and Computer Science, Gliwice, Poland
			(Very good with distinction)
		}{Mar 2016 – Sep 2017}
			Modules includes: Algorithms and Data Structures, Analysis and Design of Information Systems, Performance Evaluation of Computer Networks/Computer Systems, Software Development in a Changing Business Environment, Computer Vision and Image Recognition (MATLAB), Introduction to Compilers.

		\datedsubsection{
			BSc in Informatics: Silesian University of Technology, Faculty of Automatic Control,
			Electronics and Computer Science, Gliwice, Poland
			(Very good)
		}{Oct 2012 – Jan 2016}
			Modules includes: Programming in C/C++, Calculus and Linear Algebra, Digital Systems Arithmetic, Assembler, Statistical Methods, Discrete Mathematics, Algorithms and Data Structures, Databases,
			Software Engineering, Operating Systems (Windows 7 and Linux administration), Concurrent Computing (ADA), Computer Architecture (CUDA, JavaSpaces, PVM), Java and Programming in the Internet, Computer Graphics, Computer System Interface, Computer Networks, Rule-based Artificial Intelligence Systems.
			Individual Thesis: System for smart metering of household services using Bluetooth radio:
			Verification of possibilities to implement remote meter reading for various media in a household, based on the Bluetooth 4.0 LE standard. Modelling mesh communication system based on Texas Instruments CC2640 toolkits, for gathering real time data from meters and provide information to central system.

	\section{Skills, Interests \& Extracurricular Activities}
		Languages: Fluent English, native Polish and basic conversational French.
		IT Skills: Confident IT user. Experience in C\# and .Net, WPF, WinForms, MVC. Experience in working with Oracle, PostgreSQL, MSSQL and SQLite databases. Knowledge of scripting languages: PowerShell, bash. Basics in NodeJS. Academic experience in NVidia CUDA platform. Experience with Windows and Unix-based operating systems, especially CentOS and Ubuntu. Programming IDE: Visual Studio, IntelliJ IDEA, PyCharm, NetBeans, Vim and Atom. Familiar with Git, TFS and SVN.
		Individual Interests: Passionate about trending technologies and IoT. Loves snowboard, motorboats and martial arts. Big fan of cuisines all over the world. Participated in Night of the Living Devs hackathon for students, organised by Microsoft.
		Organisational skills: Good at brainstorming and working as part of diversified teams with Agile methodologies, including Scrum and Kanban. Familiar with Jira project management software.
		Volunteering: Ex-member of Student Microsoft .Net Group at Silesian University of Technology, responsible for the organisation of events and conferences such as IT Academics Day.

\end{document}